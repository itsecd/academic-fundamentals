\documentclass[12pt]{article}

\usepackage[english, russian]{babel}

\title{Примеры}
\author{Кипкаева Ольга Сергеевна}
\date{сегодня}

\begin{document}% \begin - окружение, которое надо закрывать с помощью \end

\maketitle % Заголовок

\section{Формулы} % \section - команда, которую не надо закрывать

Пример 1 $\alpha$\\ % не выносится из текста
Пример 2 $$\alpha_1$$ % выносится на середину строки

Пример 3 \[\alpha_2\] % выносится на середину строки

Специальные знаки $\%$\\

\section{Списки}
\subsection*{С заголовками} % раздел не пронумеровался
\begin{description}
\item[itemize:] пункты помечаются маркерами;
\item[enumerate:] пункты нумеруются;
\item[description:] пункты снабжаются заголовками.
\end{description}

\subsection{Нумерованный}
\begin{enumerate}
\item первый пункт
\item второй пункт
\end{enumerate}

\subsection{Маркированный}
\begin{itemize}
\item первый пункт
\item второй пункт
\end{itemize}

\subsection{Свой}
\begin{itemize}
\item[$+$] первый пункт
\item[$-$] второй пункт
\end{itemize}


\subsection{Вложенный}
\begin{enumerate}
\item Нумеруются
\begin{enumerate}
\item второй уровень вложенности
\end{enumerate}
\item еще один пункт
\end{enumerate}

\subsection{Ненумерованный}
\begin{trivlist}
\item первое
\item второй
\item третий
\end{trivlist}

\section{Таблички}
\begin{tabular}{|c|c||}
1 & 5 \\
\hline
2 & 3
\end{tabular}
\\
\\
\\
\begin{tabular}{|c|c|c|p{6cm}|}
\hline
1 & 5 & 6 & 9 \\
\hline
3 & 8 & 4 & 2 \\
\hline
9 & 5 & 5 & 4\\
\hline
\end{tabular}


\section{Колонки}
\begin{minipage}[t]{14mm}
Левая колонка узкая
\end{minipage}
\hfill
\begin{minipage}[t]{38mm}
Правая колонка немного шире
\end{minipage}



\end{document}